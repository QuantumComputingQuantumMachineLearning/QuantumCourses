\documentclass{scrreprt}
\usepackage[utf8]{inputenc}
\usepackage[T1]{fontenc}
\usepackage{graphicx}
\usepackage{longtable}
\usepackage{wrapfig}
\usepackage{rotating}
\usepackage[normalem]{ulem}
\usepackage{amsmath}
\usepackage{amssymb}
\usepackage{capt-of}
\usepackage{hyperref}
\usepackage[colorinlistoftodos]{todonotes}
\usepackage{comment}

\usepackage[margin=2cm]{geometry}
\usepackage{setspace}
\usepackage[standardsections]{scrhack}
\usepackage{titlesec}
\usepackage{enumitem}
\usepackage{float}
\usepackage[export]{adjustbox}
\setlength{\parindent}{1em}
\setlength{\parskip}{.3em}
\usepackage[square,sort&compress,comma,numbers]{natbib}
\bibliographystyle{apsrev}
\usepackage[T1]{fontenc}
\usepackage{eurosym}
\usepackage{graphicx}
\graphicspath{{./figures/}}
\usepackage[labelformat=empty]{caption}
\renewcommand{\thechapter}{\arabic{chapter}}
\renewcommand{\thesection}{\arabic{section}}
\renewcommand{\thesubsection}{\thesection.\arabic{subsection}}
\renewcommand{\thesubsubsection}{\thesubsection.\arabic{subsubsection}}
\renewcommand{\chaptername}{TEST}
\titlespacing{\paragraph}{0pt}{4pt}{2pt}
\titlespacing{\subsection}{0pt}{6pt}{4pt}
\titlespacing{\section}{0pt}{4pt}{8pt}
\newcommand\fpm[1]{\textcolor{blue}{#1}}
\renewcommand{\bibsection}{\section*{}}

\author{Oslo Metropolitan University}
\date{\today}
\title{}
\hypersetup{
 pdfauthor={OsloMet},
 pdftitle={},
 pdfkeywords={},
 pdfsubject={},
 pdfcreator={Emacs 29.2 (Org mode 9.7.17)},
 pdflang={English}}
\begin{document}

\section*{\LARGE{Quantum TALENT}}





\section{Relevance to the call for the proposal}

The Quantum-TALENT (\underline{T}raining and \underline{A}dvanced
\underline{L}ectures in \underline{E}mergi\underline{N}g
\underline{T}echnologies) aims at developing a consistent national training and
educational program at a graduate  level in quantum technology, quantum
computing and quantum information theory and related topics within  artificial
intelligence/machine learning (AI/ML) methods for quantum science,
which will facilitate the development of a workforce with the
competences and knowledge to meet future technological challenges and
developments. This will  be done in close collaboration with all
partners and follows up with innovative educational initiatives  to the recent evaluation of Norwegian Physics.


Quantum technologies promise to transform a wide range of industries
by delivering capabilities far beyond what is possible with classical
systems. From revolutionizing drug discovery and materials science to
enabling ultra-secure communications and advanced sensors, quantum
computing and quantum information science are poised to become pillars
of 21st-century innovation. Recognizing this potential, governments
and industries worldwide have made quantum technology a strategic
priority. For example, the United States has elevated Quantum
Information Science and Technology (QIST) to one of its highest
national research and development priorities, yet currently lacks
sufficient qualified talent to maintain competitiveness, indeed, the
shortage of quantum-trained professionals is now viewed as a national
security risk. Similarly, Europe’s \textit{Quantum Flagship}
initiative highlights that a trained workforce is crucial for powering
the emerging quantum industry and ensuring the success of the {\em
  second quantum revolution}.

Beyond the strategic and scientific imperatives, the economic stakes
of quantum technology are enormous. Recent analyses project that
quantum computing alone could create hundreds of billions of dollars
in global value in the coming decades (with forecasts of up to
$850$~billions of USD in economic impact by 2040). Nations and companies that
lead in quantum innovation stand to gain significantly in economic
competitiveness and technological advantage.

However, realizing this promise will require a robust talent
pipeline. At present, the field faces a critical skills gap: studies
find that there is only about one qualified quantum expert available
for every three job openings in the sector. Without substantial
investment in education and training, jobs requiring these skills may not have properly qualified applicants. In response to these challenges, policymakers
and scientific leaders are calling for new and sustained educational
initiatives at all levels. Developing a new generation of quantum
scientists and engineers is viewed as essential to accelerate
research, drive innovation, and maintain leadership in this strategic
field.


\section{Excellence}

\begin{comment}
{\em Provide an expanded project description. Specify and describe the individual deliverables in the researcher school (activities, coordination activities, cooperation) in accordance with the requirements set out in the call. Please describe:
-	the expected scope, including the expected number of associated PhD candidates when the school is in full operation, as well as the assignment of supervisors and other scientific personnel;
-	the researcher school's scientific content in a broad sense, and objectives and delimitations, including the balance between measures to enhance scientific quality and relevance. The school’s activities should mainly be new initiatives or the continuation of existing initiatives. It must be clearly stated what is new;
-	national and (if relevant) international research groups and other partners that will form part of the network. Their contributions must be specified;
-	how the researcher school will quality assure activities and how the results will be documented.
Explain how the researcher school is linked to or will support the
doctoral degree programme at the collaborating institutions
scientifically, strategically and in terms of its benefit to
society. It is also important to clarify how the initiative is
innovative and will lead to added value beyond the activities that are
already in place or are currently being carried out.}
\end{comment}




To address the identified workforce needs, we propose a graduate
curriculum comprising of five specialized courses covering the breadth of
quantum technologies. Each course targets a critical
sub-discipline—spanning theoretical foundations, experimental
techniques, and emerging applications to ensure that graduate students acquire
well-rounded and in-depth expertise. The courses the propose (described in detail below) are
\textit{Quantum Computing}, \textit{Quantum Information Theory}, \textit{Building Quantum Hardware},
\textit{Quantum Metrology and Sensing}, and \textit{Quantum Machine Learning}. The courses cover to a large extent also existing research directions in QIST at Norwegian universities. Several of these universities offer some courses in QIST, but mostly at the undergraduate level. Proposing an advanced portfolio of courses in QIST at the master of science and doctoral level, will then enable Norwegian  universities to develop a new and advanced set of courses which will provide our society at large with a quantum-ready work force.

The planned  courses are presently not offered at the graduate level at any
of the involved Norwegian institutions.  This program will thus provide
a unique national educational background for graduate students in
Norway and will complement educational programs at the bachelor and
master degree level at our universities. It will bring together
essentially all universities which are offering courses and degrees in
quantum technologies and as such it will also lay the foundation for a
unified national strategy on education in quantum technologies.

The program is a national collaboration between the following institutions\todo{Her boer vi vel vere konsistente om lista boer dekke dei relevante institutta eller universiteta?}

\begin{enumerate}
\item Oslo Metropolitan University, Oslo, Norway. Oslo Metropolitan University will lead the initiative.
\item Department of Physics, University of Oslo
\item Kristiania University College, Oslo, Norway
\item University of South Eastern Norway, Kongsberg, Norway
\item Department of Physics, University of Bergen, Norway
\item Department of Physics, Norwegian University of Science and Technology, Trondheim, Norway
\item SIMULA research laboratory, Oslo, Norway
\end{enumerate}


\subsection{Activities, organisation and cooperation}

The five courses we have devised reflect central topics in quantum
technologies. All courses are scaled to ten credits ECTS and will be
taught in-person and remotely over a certain number of weeks during a
regular semester. Completion of a final project is mandatory
for the final approval.  Students can participate either in person or
remotely via dedicated classrooms available at each university.  A
final in-person week for the finalization of the project is planned.
We expect to offer two courses per year. The educational material to
be developed will be available to all universities and will lay the
foundation for a national curriculum in quantum technologies. These
courses will be modularized and can be offered, in a revised format,
to participants from industry and the public sector under eventual
continuing and further education programs.

We plan to offer two courses each year, with up to thirty participants. This means that during the course of a PhD project, a student would then be able to attend courses of overlap with the his/her research topics. 
During the duration of the project, we expect thus to have been able to offer each course up to three times. 
Each iteration will improve the quality of the educational material and allow the involved universities to develop sustainable educational programs in QIST.

\subsection{Background, context and needs}

{\em The primary and secondary objectives of the researcher school must be
specified in the online grant application form.  Describe the
background for the application and explain why the researcher school
and its activities are needed. Make it clear how the project is part
of larger systems and will build upon and reinforce these
(context). It is important to document good knowledge of relevant
research areas, existing structures and activities, and the central
challenges in the field, both nationally and internationally.}




\subsection{Added value}


No such graduate program exists presently in Norway. It will therefore
serve the whole country in developing material which will prepare a
quantum-ready workforce. Furthermore, since the courses will be
properly modularized, the material can be reused and tailored to
participants outside academia, such as public and private enterprises
which deal with quantum technologies. Courses which target users from
the private and public sectors outside of academia will fall under the
auspices of national continuing and further education programs (EVU
programs in Norwegian) and will be developed separately in close
collaboration with external partners.

\section{Impact}


The short term impact is to make education and research in QIST
visible to a broader community of potential students. Developing such a homogeneous educational program across Norwegian universities will provide in the long term a quantum-ready workforce. Since many universities do not have the capacity and knowledge to offer all these topics, a national collaboration can thus aid in developing education material which can be tailored to local research and educational plans, and thereby strengthen national efforts in QIST.

\subsection{Communication, dissemination and target groups}

All educational material will be freely available to students from all
Norwegian universities. The material will be shared by teachers from
the different universities, allowing thereby for a continuous upgrade
of the educational material.  To achieve this, all courses, with
source material such as jupyter-notebooks, lecture notes and more,
will be available via repositories like GitHub or GitLab.


\subsection{Relevance to society}
As stated above, quantum technologies are expected to play a significant role, now and in the coming years. Preparing the ground for a quantum-ready workforce is thus of strategic interests for Norwegian industry and the society as a whole. 
The educational material will at a later stage be tailored for short intensive courses which will be offered to participants from the both the public and private sectors. 

\subsection{Environmental impact, ethical perspectives, recruitment of women/gender balance }

We do not foresee any negative enviromental impact (quantum technologies hold promise of reducing energy consuption). In this program, our focus is on ethical perspectives and gender balance.
\todo{To the extent that students are supposed to gather, physically, to finish their joint projects, some traveling would be necessary. As this would have a carbon dioxide imprint, I guess we should address -- and justify -- this.}

\subsubsection{Ethical perspectives}
Quantum technologies have far-reaching ethical and societal implications that must be addressed in our  graduate program. Key concerns we will address include


\begin{itemize}
\item Privacy and data security: The immense processing power of quantum computers could break current encryption, enabling unauthorized access to sensitive data. As quantum systems handle vast amounts of personal information, questions arise over data ownership and control, with risks of losing control over personal data and heightened privacy concerns. Such capabilities may also facilitate mass surveillance through advanced quantum sensing and computing, potentially enabling new forms of monitoring that threaten civil liberties.
\item Military and defense use: Quantum technology is a dual-use domain with significant military applications. Quantum sensors can detect hidden objects or submarines, and quantum decision-making systems might outpace human response. The prospect of autonomous or rapid quantum-driven military decisions raises profound ethical questions about oversight and accountability in warfare. An arms race in quantum capabilities could also destabilize global security, as nations rush to exploit quantum advantages in communications and computing.
\item Societal impact and equity: The deployment of quantum technologies could disrupt economies and labor markets. For example, quantum automation and algorithms might displace certain jobs, echoing broader concerns of technological unemployment. There is also a risk that access to quantum computing will be unequal – if only a select few organizations or countries control the technology, existing digital divides and social inequalities may worsen. Ensuring equitable access and considering the global societal impact of quantum advancements are therefore ethical aspects we will include in the discussions of the various courses.
\end{itemize}



These implications highlight why ethics must be a foundational
component of quantum technology education. Our graduate program will
prepare students to foresee and responsibly manage the consequences of
quantum innovations on society, security, and human rights.


\subsubsection{Recruitment of women and gender balance}

Promoting gender balance is another critical component of a
responsible and inclusive quantum technology program. Women and other
gender minorities remain starkly underrepresented in quantum science
and engineering. In fact, recent industry data show the quantum sector
is heavily male-dominated – fewer than $10\%$ of applicants for quantum
technology jobs are female. This imbalance signals an urgent need for
educational strategies that encourage more women to enter and succeed
in quantum fields. Beyond equity concerns, diverse representation is
vital for the health of the discipline itself. Gender diversity in
quantum computing is not just a moral imperative – it is  a strategic one. 
Studies consistently find that diverse teams are more innovative and
produce better results, benefiting scientific creativity and
problem-solving. By including more women in quantum research, the field
gains a wider range of perspectives and approaches, ultimately driving
more robust and inclusive technological solutions.


By highlighting the collaborative nature of quantum research and the
successes of women in the field, our program can help change the
narrative that quantum is an exclusively male domain. Visible role
models are especially powerful: the presence of women leaders in
quantum computing inspires the next generation and provides concrete
examples that young women can aspire to . Ultimately, prioritizing
gender balance in quantum education is about ensuring fairness,
unlocking innovation, and creating a field that welcomes all excellent
minds.

In order to address many of these points and promote equal
participation, our graduate program in quantum technology will adopt
strategies that create an inclusive and supportive educational
environment. Key practices include we will address are:

\begin{itemize}
\item Inclusive Pedagogy: Instructors should employ teaching methods that actively engage and support students of all genders and backgrounds.
\item Mentorship and Role Models: Establishing mentorship programs has the potential to  significantly support women in quantum technology pathways. Connecting female graduate students with experienced mentors – faculty, alumni, or industry professionals – provides guidance, networking, and moral support. 
\item Outreach and Recruitment Initiatives: A long-term strategy for gender balance is to strengthen the pipeline of women entering quantum studies. 
\item Supportive Community and Policies: Fostering an equitable environment also means establishing a culture that actively supports diversity and inclusion. 
\end{itemize}



\section{Implementation}

These courses address many central concepts, such as quantum
mechanical superposition, entanglement, quantum technology applications, and many
different methods and algorithms for AI and machine learning, covering
both supervised and unsupervised learning as well as central
discriminative and generative deep learning methods. Programming is
indispensable in all courses and course participants learn to
study complicated problems which require knowledge and skills
necessary for educating a modern workforce. Many of these courses
focus also on teamwork, project management, and communication.

The courses are
\begin{enumerate}
\item \textit{Quantum Computing},
\item \textit{Quantum Information Theory},
\item \textit{Building Quantum Hardware},
\item \textit{Quantum Metrology and Sensing}, and
\item \textit{Quantum Machine Learning}.
\end{enumerate}

For each course, we provide a concise description, the intended learning outcomes, and the key topics to be covered.
All courses correspond to 10 credits (ECTS).

\subsection{Course 1: Quantum Computing, 10 ECTS}

\textbf{Description:} This course provides a rigorous introduction to the principles and algorithms of quantum computing. Students will learn how information can be encoded in quantum bits (qubits) and processed using quantum logic operations that exploit phenomena such as superposition and entanglement. The curriculum covers the design and analysis of quantum circuits and the most important quantum algorithms that demonstrate computational advantages over classical approaches. Emphasis is placed on both the theoretical foundations (e.g., the quantum circuit model and complexity theory) and practical aspects, including the use of quantum programming frameworks to implement algorithms on simulators or prototype quantum hardware.

\textbf{Learning Outcomes:}
\begin{itemize}
\item Understand the fundamental concepts of quantum computation, including qubits, superposition, entanglement, and the postulates of quantum mechanics relevant to computing.
\item Design and analyze quantum logic circuits using standard quantum gates (Hadamard, Pauli-X/Y/Z, CNOT, etc.) to carry out computational tasks and simple algorithms.
\item Explain and apply major quantum algorithms such as Shor’s factoring algorithm and Grover’s search algorithm, and evaluate their complexity and speedups relative to classical algorithms for the same problems.
\item Compare quantum and classical computational power and complexity (e.g., understanding the \textit{bounded-error quantum polynomial time}-class (BQP) and how it relates to classical complexity classes), articulating the theoretical capabilities and limits of quantum computers.
\item Describe the principles of quantum error correction and discuss the impact of decoherence and noise on quantum computations, along with basic strategies for error mitigation and fault tolerance.
\item Develop and run simple quantum programs using contemporary quantum software tools (such as Qiskit or Cirq), gaining hands-on experience with implementing quantum circuits and observing their outputs.
\end{itemize}

\textbf{Key Topics:}
\begin{itemize}
\item Qubits and quantum states; representation of information in two-level quantum systems (Bloch sphere, state vectors, Dirac notation).
\item Quantum logic gates and the circuit model of computation; universal gate sets and circuit complexity.
\item Representative quantum algorithms (Deutsch-Jozsa, Bernstein-Vazirani, Grover’s search, Shor’s factoring, Quantum Fourier Transform) and their applications.
\item Basics of quantum computational complexity theory (notions of oracle algorithms, BQP vs. classical complexity classes, quantum supremacy concept).
\item Introduction to quantum error correction (simple codes like bit-flip and phase-flip codes, Shor’s code) and the effects of decoherence on computational reliability.
\item Practical quantum programming and experimentation using simulators or cloud-based quantum processors (writing and executing small quantum circuits).
\end{itemize}

\subsection{Course 2: Quantum Information Theory, 10 ECTS}

\textbf{Description:} This course delves into the theoretical foundations of information science in the quantum domain. It examines how information is quantified, transmitted, and protected when using quantum states in place of classical bits. Students will study the quantum analogues of classical information concepts and learn how quantum mechanics enables new communication and cryptographic protocols that have no classical counterpart. Key topics include quantum entropy and information measures, the role of entanglement as an informational resource, quantum communication protocols like teleportation and superdense coding, and the basics of quantum cryptography and error correction. Through these topics, students will gain a deeper understanding of the limits and possibilities of information processing in quantum systems.

\textbf{Learning Outcomes:}
\begin{itemize}
\item Articulate the differences between classical and quantum information, including concepts such as qubits vs. bits, the no-cloning theorem, quantum vs. classical randomness, and the implications of measurement on information.
\item Use the mathematical formalism of quantum information (state vectors, density operators, trace, partial trace) to describe quantum states (pure and mixed) and calculate information-theoretic quantities like von Neumann entropy and mutual information.
\item Explain fundamental quantum communication protocols (e.g., quantum teleportation and superdense coding) step-by-step and analyze how they utilize entanglement and quantum measurements to achieve tasks impossible or inefficient with classical information alone.
\item Understand the principles of quantum cryptography, especially quantum key distribution (for example, the BB84 protocol), and be able to discuss why quantum approaches can guarantee security based on the laws of physics.
\item Describe how quantum noise and decoherence are modeled as quantum channels, and outline how quantum error-correcting codes (such as the Shor code or simple stabilizer codes)\todo{This point is very similar to one included in the QC course. But I suppose some redudancy is fine.} can protect quantum information from errors.
\item Apply quantum information theory concepts to evaluate the capacity of simple quantum channels and interpret limits like Holevo’s bound that quantify the maximum classical information that can be extracted from quantum states.
\end{itemize}

\textbf{Key Topics:}
\begin{itemize}
\item Quantum state representation and information: qubits vs. classical bits, bra-ket notation, density matrices for mixed states, quantum entropy.
\item Entanglement and non-local correlations: Bell states, EPR pairs, measures of entanglement, monogamy of entanglement.
\item Quantum information measures: von Neumann entropy, conditional entropy, mutual information, Holevo’s bound on accessible information.
\item Quantum communication protocols: quantum teleportation, superdense coding, entanglement swapping, quantum repeaters (conceptual introduction).
\item Quantum cryptography: quantum key distribution (BB84, Ekert protocol), no-cloning theorem and its implications for security, quantum random number generation.
\item Quantum error correction and quantum channels: noise models (bit-flip, phase-flip, depolarizing channel), simple quantum error-correcting codes, channel capacity concepts (classical capacity of a quantum channel, quantum capacity basics).
\end{itemize}

\subsection{Course 3: Building Quantum Hardware, 10 ECTS}

\textbf{Description:} This course explores the physical realization of
quantum computers and other quantum information processing
devices. Students will learn about the leading hardware platforms for
qubits—such as superconducting circuits, trapped ions, semiconductor
quantum dots, photonic qubits, and defect centers—and the fundamental
physics that enables each platform. The curriculum examines how qubits
are initialized, controlled (manipulated with pulses or fields), and
read out, as well as the engineering challenges involved in scaling up
these systems. Key considerations include maintaining coherence
(isolating qubits from environmental noise), error rates and fidelity
of quantum operations, cryogenic and vacuum requirements, and
approaches to integrate many qubits into a single functioning quantum
processor. By surveying current state-of-the-art quantum devices and
experimental techniques, students will gain insight into how to
design, operate, and evaluate quantum hardware.

\textbf{Learning Outcomes:}
\begin{itemize}
\item Explain the operating principles of major qubit implementation technologies (e.g., how superconducting qubits use Josephson junctions to create quantized energy levels, how trapped ions use electromagnetic fields to trap and manipulate ion qubits, how photon-based qubits encode information in light modes, etc.).
\item Describe the methods for qubit control and readout in different platforms, including how quantum gates are physically realized (e.g., microwave pulses for superconducting qubits, laser pulses for ion qubits, optical interferometers for photonic qubits) and how qubit states are measured.
\item Identify the primary sources of noise and decoherence in quantum hardware (such as spontaneous emission, dephasing from material defects, thermal noise, crosstalk between qubits) and discuss strategies to mitigate these effects (improved materials, shielding, error suppression techniques).
\item Understand the engineering infrastructure required for quantum hardware: for instance, the use of dilution refrigerators for superconducting devices, ultra-high vacuum and laser cooling systems for ion traps, nanofabrication techniques for solid-state qubits, and high-speed electronics for control and readout.
\item Evaluate the challenges and current approaches in scaling quantum processors from a few qubits to many qubits, including issues of inter-qubit connectivity, control signal delivery, fabrication scalability, and modular architectures. Discuss how these challenges are being addressed through techniques like multiplexing, error correction (logical qubits), and networked quantum modules.
\item Interpret key performance metrics of quantum hardware from technical literature or data sheets (coherence times $T_1$ and $T_2$, gate and readout fidelities, two-qubit gate error rates, qubit counts) and assess what they imply for the capability of the device to run certain algorithms or maintain entanglement over time.
\end{itemize}

\textbf{Key Topics:}
\begin{itemize}
\item Qubit physical platforms: superconducting qubits (transmons, flux qubits) and circuit QED; trapped-ion qubits (linear RF traps, Penning traps); spin qubits in semiconductors (quantum dots, donor spins); photonic qubits (single photons, integrated photonic circuits); nitrogen vacancy (NV) centers and other solid-state defect qubits.
\item Qubit control and readout mechanisms: microwave control of superconducting qubits, laser manipulation of atomic qubits, optical measurement techniques (homodyne detection, single-photon detectors), and classical hardware for signal generation and detection.
\item Coherence and noise: sources of decoherence (phase noise, energy relaxation), quantitative measures of coherence (coherence time, fidelity), error rates and how they scale with system size.
\item Cryogenics and vacuum technology: dilution refrigerator operation for millikelvin temperatures, cryostats, ultra-high vacuum systems for ion and neutral atom devices.
\item Scalability and integration: 2D and 3D qubit integration, on-chip wiring and control lines, cross-talk management, emerging technologies for scaling (such as photonic interconnects between quantum modules, cryogenic CMOS control electronics).
\item Hardware-level error correction and mitigation: physical versus logical qubits, implementation of error-correcting codes in hardware (e.g., layout for the surface code, syndrome extraction circuitry), and trade-offs in resource overhead.
\item Overview of current quantum processors and roadmaps: examples of quantum chip architectures from industry or research (e.g., superconducting processors with 50+ qubits, ion trap systems with shuttling ions), and discussion of performance milestones (quantum volume, error rates approaching fault-tolerance thresholds).
\end{itemize}

\subsection{Course 4: Quantum Metrology and Sensing, 10 ECTS}

\textbf{Description:} This course focuses on leveraging quantum phenomena to achieve ultra-precise measurements and sensing capabilities beyond classical limits. Students will examine how quantum mechanical effects—such as superposition, entanglement, and squeezing—can be utilized to enhance the sensitivity of measurements of time, fields, forces, and other physical quantities. The theoretical component of the course introduces the framework of quantum parameter estimation and the concept of measurement uncertainty in quantum mechanics (including the standard quantum limit and the ultimate Heisenberg limit of precision). On the practical side, the course surveys a range of cutting-edge quantum sensors and metrological devices: atomic clocks that keep time with unprecedented accuracy, interferometric sensors (like those used in gravitational wave detectors) that incorporate quantum light to improve sensitivity, magnetometers and accelerometers using quantum spins or superconducting circuits, and other emerging quantum sensing applications. By the end of the course, students will understand how quantum information concepts translate into superior measurement performance and will appreciate both the potential and the challenges of deploying quantum sensors in real-world scenarios.

\textbf{Learning Outcomes:}
\begin{itemize}
\item Explain the quantum limits of measurement precision, distinguishing between the standard quantum limit (SQL) that often arises from shot noise in classical measurements and the Heisenberg limit achievable with quantum resources (scaling of uncertainty inversely with the number of particles or quanta).
\item Apply principles of quantum estimation theory (such as calculating the Fisher information and employing the Cramér-Rao bound) to determine the best achievable sensitivity for estimating a physical parameter in a given quantum measurement setup.
\item Describe how specific non-classical states of light or matter can improve sensing performance: for example, how squeezed light reduces noise in one quadrature to improve interferometer sensitivity, or how entangled atoms in an atomic clock can surpass the precision of independent atoms.
\item Provide examples of advanced quantum sensor systems and discuss their operating principles: including atomic clocks (using superposition in atomic transitions for time standards), quantum optical interferometers for gravitational wave detection (using squeezed vacuum states to lower noise), quantum magnetometers (e.g., superconducting quantum interference devices (SQUIDs) or spin-based sensors like NV centers in diamond, measuring extremely small magnetic fields), and inertial sensors (quantum accelerometers/gyroscopes using matter-wave interference).
\item Analyze practical challenges in quantum metrology experiments—such as loss of coherence in entangled states, photon losses in optical systems, or technical noise—and explain methods to address them (for instance, error correction techniques adapted to metrology, active stabilization and feedback control, or entanglement distillation to counteract noise).
\item Design or evaluate a conceptual quantum sensing experiment for a given application, identifying what quantum resources (entanglement, squeezing, etc.) are required to reach a target sensitivity and discussing how one would implement and measure those resources in practice.
\end{itemize}

\textbf{Key Topics:}
\begin{itemize}
\item Quantum measurement and uncertainty: operators and observables, projective measurements vs. POVMs, uncertainty relations and their implications for measurement precision.
\item The standard quantum limit (SQL) versus the Heisenberg limit in various contexts (phase estimation, frequency estimation); quantum Fisher information and the basics of the quantum Cramér-Rao bound.
\item Squeezed states and quantum noise reduction: generation of squeezed light (optical parametric oscillators), use of squeezed vacuum in interferometers (e.g., LIGO) to improve signal-to-noise.
\item Entangled states for metrology: GHZ (Greenberger–Horne–Zeilinger) states, NOON states for interferometry, spin-squeezed states in atomic ensembles, and how these achieve sub-SQL performance.
\item Quantum sensor technologies: atomic clocks (cesium fountain clocks, optical lattice clocks), quantum magnetometers (SQUIDs, atomic magnetometers, NV-center magnetometry), quantum accelerometers and gyroscopes (atom interferometry for inertial navigation), and quantum imaging techniques (quantum illumination, ghost imaging concepts).
\item Noise and error management in sensing: effects of decoherence on sensor performance, photon loss, phase noise; introduction to error mitigation and error correction strategies in the context of metrology (e.g., using entangled states that are robust to certain noise, or feedback stabilization).
\item Case studies in quantum metrology: e.g., the role of squeezed light in gravitational wave detectors, demonstrations of entanglement-enhanced clock precision, and applications of quantum sensing in domains like biomedical imaging or geoscience.
\end{itemize}

\subsection{Course 5: Quantum Machine Learning, 10 ECTS}

\textbf{Description:} This course examines the intersection of quantum computing and machine learning, exploring how quantum computers could potentially accelerate machine learning tasks and how concepts from machine learning can aid quantum computing research. Students will be introduced to the algorithms and frameworks of quantum machine learning (QML), learning how classical data can be embedded into quantum states and processed through quantum circuits. The course covers quantum-enhanced versions of various learning algorithms—for instance, quantum support vector machines that utilize quantum kernels, quantum algorithms for principal component analysis and linear systems solving, and quantum neural network models implemented as parameterized quantum circuits. It also addresses hybrid quantum-classical approaches (such as variational algorithms) that are central to near-term QML applications. Throughout, an emphasis is placed on understanding both the potential speedups and advantages that quantum computing offers for machine learning, and the practical limitations (hardware noise, qubit constraints, training difficulties) that currently influence this emerging field. Students will gain experience with QML libraries and simulators, preparing them to contribute to research or applications in quantum data science.

\textbf{Learning Outcomes:}
\begin{itemize}
\item Understand the fundamental concepts of quantum machine learning, including how quantum states and operations can represent and manipulate data, and how this differs from classical machine learning paradigms.
\item Demonstrate how to encode classical data into quantum states using various embedding strategies (such as basis encoding, amplitude encoding, or more complex feature map encodings), and discuss the impact of these choices on the performance of a quantum learning algorithm.
\item Implement basic quantum machine learning algorithms using available software frameworks: for example, constructing a simple quantum classifier (a quantum circuit with tunable parameters) and training it on a small dataset using a simulator or quantum cloud service.
\item Explain key algorithms that underpin potential quantum advantages for certain computational tasks relevant to machine learning—such as the Harrow–Hassidim–Lloyd (HHL) algorithm for solving linear systems (with implications for regression or classification), quantum principal component analysis for data dimensionality reduction, and Grover’s algorithm for accelerating unstructured search—and analyze their complexity compared to classical counterparts.
\item Understand and apply the concept of variational quantum algorithms (like the Variational Quantum Eigensolver and the Quantum Approximate Optimization Algorithm) in machine learning contexts, such as using a hybrid quantum-classical approach to optimize a model or a cost function for classification or clustering.
\item Critically evaluate the current state of quantum machine learning by reviewing recent literature: identify which scenarios have demonstrated or predict a quantum advantage in learning tasks and articulate the challenges (noise, scalability of training, limited qubit counts) that must be overcome for QML to become practically useful.
\end{itemize}

\textbf{Key Topics:}
\begin{itemize}
\item Data encoding and quantum feature maps: methods to input classical data into quantum computers (one-hot/basis encoding, amplitude encoding, parameterized feature maps) and how these define a feature space for quantum models.
\item Quantum linear algebra subroutines for ML: quantum algorithms for solving linear systems (the HHL algorithm), eigen-decomposition (quantum principal component analysis), and their relevance to machine learning applications.
\item Quantum-enhanced learning models: quantum support vector machines (using quantum kernel estimation), quantum clustering and nearest-neighbor algorithms, and quantum recommendation system approaches.
\item Quantum neural networks and parametrized quantum circuits: designing quantum circuits with tunable parameters as analogues to neural network models, training these circuits (e.g., via gradient descent or other optimizers), and considerations like barren plateaus in the training landscape.
\item Hybrid quantum-classical algorithms in ML: The \textit{variational quantum eigensolver} (VQE) and \textit{quantum approximate optimizatoin algorithm} (QAOA) as examples of using quantum subroutines within classical optimization loops; applications of these hybrid methods to optimization problems and supervised learning tasks.
\item QML software frameworks and tools: introduction to libraries such as TensorFlow Quantum, PennyLane, or Qiskit Machine Learning, which enable the simulation and integration of quantum models into classical machine learning workflows.
\item Emerging applications and case studies: discussion of recent demonstrations and research in QML (e.g., quantum-enhanced data classification in chemistry or finance), and an assessment of what future advances are needed to achieve real-world impact.
\end{itemize}

\subsection{Strategy clarification}

\subsection{Resources, expertise, distribution of roles and cooperation}

\subsection{Project manager and project group}
The project will be lead by Oslo Metropolitan University. The project group 


\subsection{Budget}

We apply for 16 MNOK over eight years. This corresponds to an annual budget of approximately 2 MNOK per year. More than half of the budget will be used to host students from other institutions for a one-week intensive meeting to discuss exercises an final projects. Each course will run at one of the involved universities. We estimate that approximately 30 students will attend and many of these (20 or more) will come from the other universities. With two courses we estimate the lodging, travel and per diem expenses to be approximately 1 MNOK. The remaining funds will be used to cover expenses related to the development of educational material.

\subsection{Risk}

This is a no-risk application as the involved universities have a
strong research and educational background in quantum science. Several
of the involved universities offer already one-semester courses with
similar content, but often tailored to undergraduate students. Developing and tailoring the educational material to
a national graduate program is thus seen as a very low-risk
initiative. The universities are also well-equipped with dedicated lecture rooms for remote teaching.



\subsubsection*{BIBLIOGRAPHY}
\fontsize{10}{10}\selectfont{ \setlength{\bibsep}{1pt plus 0.2ex}
\bibliographystyle{apsrev}
\renewcommand{\bibsection}{\section*{}}
%\bibliography{qML20250301Zotero}
\end{document}





