\documentclass{scrreprt}
\usepackage[utf8]{inputenc}
\usepackage[T1]{fontenc}
\usepackage{graphicx}
\usepackage{longtable}
\usepackage{wrapfig}
\usepackage{rotating}
\usepackage[normalem]{ulem}
\usepackage{amsmath}
\usepackage{amssymb}
\usepackage{capt-of}
\usepackage{hyperref}
\usepackage[margin=2cm]{geometry}
\usepackage{setspace}
\usepackage[standardsections]{scrhack}
\usepackage{titlesec}
\usepackage{enumitem}
\usepackage{float}
\usepackage[export]{adjustbox}
\setlength{\parindent}{1em}
\setlength{\parskip}{.3em}
\usepackage[square,sort&compress,comma,numbers]{natbib}
\bibliographystyle{apsrev}
\usepackage[T1]{fontenc}
\usepackage{eurosym}
\usepackage{graphicx}
\graphicspath{{./figures/}}
\usepackage[labelformat=empty]{caption}
\renewcommand{\thechapter}{\arabic{chapter}}
\renewcommand{\thesection}{\arabic{section}}
\renewcommand{\thesubsection}{\thesection.\arabic{subsection}}
\renewcommand{\thesubsubsection}{\thesubsection.\arabic{subsubsection}}
\renewcommand{\chaptername}{TEST}
\titlespacing{\paragraph}{0pt}{4pt}{2pt}
\titlespacing{\subsection}{0pt}{6pt}{4pt}
\titlespacing{\section}{0pt}{4pt}{8pt}
\newcommand\fpm[1]{\textcolor{blue}{#1}}
\renewcommand{\bibsection}{\section*{}}

\author{Oslo Metropolitan University}
\date{\today}
\title{}
\hypersetup{
 pdfauthor={OsloMet},
 pdftitle={},
 pdfkeywords={},
 pdfsubject={},
 pdfcreator={Emacs 29.2 (Org mode 9.7.17)},
 pdflang={English}}
\begin{document}

\section*{\LARGE{Quantum TALENT}}





\section{Relevance to the call for the proposal}

The Quantum-TALENT (\underline{T}raining and \underline{A}dvanced
\underline{L}ectures in \underline{E}mergi\underline{N}g
\underline{T}echnologies) aims at developing a consistent national training and
educational program at a graduate  level in quantum technology, quantum
computing and quantum information theory and related topics within  artificial
intelligence/machine learning (AI/ML) methods for quantum science,
which will facilitate the development of a workforce with the
competences and knowledge to meet future technological challenges and
developments. This will  be done in close collaboration with all
partners and follows up with innovative educational initiatives  to the recent evaluation of Norwegian Physics.


Quantum technologies promise to transform a wide range of industries
by delivering capabilities far beyond what is possible with classical
systems. From revolutionizing drug discovery and materials science to
enabling ultra-secure communications and advanced sensors, quantum
computing and quantum information science are poised to become pillars
of 21st-century innovation. Recognizing this potential, governments
and industries worldwide have made quantum technology a strategic
priority. For example, the United States has elevated Quantum
Information Science and Technology (QIST) to one of its highest
national reearch and development  priorities, yet currently lacks sufficient qualified
talent to maintain competitiveness, indeed, the shortage of
quantum-trained professionals is now viewed as a national security
risk. Similarly, Europe’s \textit{Quantum Flagship} initiative
highlights that a trained workforce is crucial for powering
the emerging quantum industry and ensuring the success of the {\em second
quantum revolution}.

Beyond the strategic and scientific imperatives, the economic stakes
of quantum technology are enormous. Recent analyses project that
quantum computing alone could create hundreds of billions of dollars
in global value in the coming decades (with forecasts of up to
$850~billion in economic impact by 2040). Nations and companies that
lead in quantum innovation stand to gain significantly in economic
competitiveness and technological advantage.

However, realizing this promise will require a robust talent
pipeline. At present, the field faces a critical skills gap: studies
find that there is only about one qualified quantum expert available
for every three job openings in the sector. Without substantial
investment in education and training, it is projected that by 2025
fewer than half of quantum computing job vacancies could be filled
under current trends. In response to these challenges, policymakers
and scientific leaders are calling for new and sustained educational
initiatives at all levels. Developing a new generation of quantum
scientists and engineers is viewed as essential to accelerate
research, drive innovation, and maintain leadership in this strategic
field.


\section{Excellence}

To address the identified workforce needs, we propose a graduate
curriculum comprising of five specialized courses covering the breadth of
quantum technologies. Each course targets a critical
sub-discipline—spanning theoretical foundations, experimental
techniques, and emerging applications to ensure that graduate students acquire
well-rounded and in-depth expertise. The courses are:
\begin{enumerate}
\item \textit{Quantum Computing},
\item \textit{Quantum Information Theory},
\item \textit{Quantum Hardware},
\item \textit{Quantum Metrology and Sensing}, and
\item \textit{Quantum Machine Learning}.
\end{enumerate}

For each course, we provide a concise description, the intended learning outcomes, and the key topics to be covered.

This program will provide a unique national educational background for
graduate students in Norway and will complement educational programs
at the bachelor and master degree level at our universities. It will bring together essentially all universities which are offering courses and degrees in quantum technologies and as such it will also lay the foundation for 

\subsection{Activities, organisation and cooperation}

\begin{enumerate}

\item Oslo Metropolitan University, Oslo, Norway

\item Department of Physics, University of Oslo

\item Kristiania University College, Oslo, Norway

\item University of South Eastern Norway, Kongsberg, Norway


\item Department of Physics, University of Bergen, Norway

\item Department of Physics, Norwegian University of Science and Technology, Trondheim, Norway

\item SIMULA research laboratory, Oslo, Norway

\end{enumerate}


Provide an expanded project description. Specify and describe the individual deliverables in the researcher school (activities, coordination activities, cooperation) in accordance with the requirements set out in the call. Please describe:
-	the expected scope, including the expected number of associated PhD candidates when the school is in full operation, as well as the assignment of supervisors and other scientific personnel;
-	the researcher school's scientific content in a broad sense, and objectives and delimitations, including the balance between measures to enhance scientific quality and relevance. The school’s activities should mainly be new initiatives or the continuation of existing initiatives. It must be clearly stated what is new;
-	national and (if relevant) international research groups and other partners that will form part of the network. Their contributions must be specified;
-	how the researcher school will quality assure activities and how the results will be documented.


Explain how the researcher school is linked to or will support the
doctoral degree programme at the collaborating institutions
scientifically, strategically and in terms of its benefit to
society. It is also important to clarify how the initiative is
innovative and will lead to added value beyond the activities that are
already in place or are currently being carried out.




\subsection{Activities, organization and cooperation}


\subsection{Background, context and needs}

{\em The primary and secondary objectives of the researcher school must be
specified in the online grant application form.  Describe the
background for the application and explain why the researcher school
and its activities are needed. Make it clear how the project is part
of larger systems and will build upon and reinforce these
(context). It is important to document good knowledge of relevant
research areas, existing structures and activities, and the central
challenges in the field, both nationally and internationally.}




\subsection{Added value}

{\em Explain how the researcher school is linked to or will support the
doctoral degree programme at the collaborating institutions
scientifically, strategically and in terms of its benefit to
society. It is also important to clarify how the initiative is
innovative and will lead to added value beyond the activities that are
already in place or are currently being carried out.}

\section{Impact}


{\em Describe the impacts and outcomes expected from the researcher school
in both the short-term and the medium-term, and the
societal/system-related impacts the school will help to generate in a
longer perspective.}

\subsection{Communication, dissemination and target groups}

{\em Specific plans for communication, including advertisement/marketing,
target groups, relevant user groups and how these will be involved in
the researcher school, which channels are to be used etc. is to be
provided in the online application form.}

\subsection{Relevance to society}


\subsection{Environmental impact, ethical perspectives, recruitment of women/gender balance }

{\em Briefly describe whether implementing the researcher school and/or applying the results will have a significant positive or negative environmental impact. Clarify whether there are any ethical issues relating to implementation of the researcher school. If there are, briefly describe how these will be dealt with. Provide a brief explanation of how the researcher school will promote the Research Council’s general objectives to increase the recruitment of women in research and improve the gender balance.}


\subsubsection{Ethical perspectives}
Quantum technologies have far-reaching ethical and societal implications that must be addressed in our  graduate program. Key concerns we will address include

\begin{itemize}
\item Privacy and data security: The immense processing power of quantum computers could break current encryption, enabling unauthorized access to sensitive data. As quantum systems handle vast amounts of personal information, questions arise over data ownership and control, with risks of losing control over personal data and heightened privacy concerns. Such capabilities may also facilitate mass surveillance through advanced quantum sensing and computing, potentially enabling new forms of monitoring that threaten civil liberties.
\item Military and defense use: Quantum technology is a dual-use domain with significant military applications. Quantum sensors can detect hidden objects or submarines, and quantum decision-making systems might outpace human response. The prospect of autonomous or rapid quantum-driven military decisions raises profound ethical questions about oversight and accountability in warfare . An arms race in quantum capabilities could also destabilize global security, as nations rush to exploit quantum advantages in communications and computing.
\item Societal impact and equity: The deployment of quantum technologies could disrupt economies and labor markets. For example, quantum automation and algorithms might displace certain jobs, echoing broader concerns of technological unemployment . There is also a risk that access to quantum computing will be unequal – if only a select few organizations or countries control the technology, existing digital divides and social inequalities may worsen . Ensuring equitable access and considering the global societal impact of quantum advancements are therefore ethical aspects we will include in the discussions of the various courses..
\end{itemize}



These implications highlight why ethics must be a foundational
component of quantum technology education. Our graduate program will
prepare students to foresee and responsibly manage the consequences of
quantum innovations on society, security, and human rights.


\subsubsection{Recruitment of women and gender balance}

Promoting gender balance is another critical component of a
responsible and inclusive quantum technology program. Women and other
gender minorities remain starkly underrepresented in quantum science
and engineering. In fact, recent industry data show the quantum sector
is heavily male-dominated – fewer than $10\%$ of applicants for quantum
technology jobs are female. This imbalance signals an urgent need for
educational strategies that encourage more women to enter and succeed
in quantum fields. Beyond equity concerns, diverse representation is
vital for the health of the discipline itself. Gender diversity in
quantum computing isn’t just a moral imperative – it is  a strategic one
. Studies consistently find that diverse teams are more innovative and
produce better results, benefiting scientific creativity and
problem-solving. By including more women in quantum research, the field
gains a wider range of perspectives and approaches, ultimately driving
more robust and inclusive technological solutions.


From an ethical standpoint, a lack of diversity can also skew the
development of technology. Homogeneous teams may inadvertently encode
biases or blind spots into quantum algorithms and applications
. Ensuring women participate equally in quantum research and education
helps to broaden the vision of what problems are prioritized and who
benefits from breakthroughs. It fosters technology that serves a wider
populace and avoids “one-size-fits-all” outcomes. Moreover, providing
equitable opportunities in cutting-edge fields like quantum computing
is a matter of social justice and talent utilization – there is
abundant untapped talent among women in STEM, and inspiring more women
to pursue quantum careers will enlarge the overall talent pool driving
this revolution .



Addressing gender disparities requires tackling both pipeline issues
and the culture within graduate programs. Many young women are
dissuaded from pursuing quantum science due to persistent stereotypes
and environmental factors. For example, there is often a perception
that one must fit a “cutthroat and impersonal” mold to succeed in
intense STEM research, which can discourage those who don’t identify
with that stereotype . In reality, effective research in quantum
science – as in any science – thrives on collaboration, creativity,
and diversity of thought, not on any single personality type
. Graduate programs should actively work to dispel myths about who can
excel in quantum technology. By highlighting the collaborative nature
of quantum research and the successes of women in the field, educators
can help change the narrative that quantum is an exclusively male or
“genius” domain. Visible role models are especially powerful: the
presence of women leaders in quantum computing inspires the next
generation and provides concrete examples that young women can aspire
to . Ultimately, prioritizing gender balance in quantum education is
about ensuring fairness, unlocking innovation, and creating a field
that welcomes all excellent minds.

In order to address many of these points and promote equal
participation, our graduate program in quantum technology will adopt
strategies that create an inclusive and supportive educational
environment. Key practices include we will address are:

\begin{\itemize}

\item Inclusive Pedagogy: Instructors should employ teaching methods that actively engage and support students of all genders and backgrounds.
\item Mentorship and Role Models: Establishing mentorship programs has the potential to  significantly support women in quantum technology pathways. Connecting female graduate students with experienced mentors – faculty, alumni, or industry professionals – provides guidance, networking, and moral support. 
\item Outreach and Recruitment Initiatives: A long-term strategy for gender balance is to strengthen the pipeline of women entering quantum studies. 
\item Supportive Community and Policies: Fostering an equitable environment also means establishing a culture that actively supports diversity and inclusion. 
\end{itemize}



\section{Implementation}

These courses address many central concepts, such as quantum
mechanical superposition, entanglement, QT applications, and many
different methods and algorithms for AI and machine learning, covering
both supervised and unsupervised learning as well as central
discriminative and generative deep learning methods. Programming is
indispensable in all courses and course participants learn to
study complicated problems which require knowledge and skills
necessary for educating a modern workforce. Many of these courses
focus also on teamwork, project management, and communication.



\begin{enumerate}
\item Review of Quantum physics, with elements from linear algebra and statistics
\item Basic introduction to QT, QIT and QC. All involved universities offer similar courses
\item Quantum mechanics (only at UiO as of now)
\item Quantum Computing and central algorithms, gates, circuits and more
\item Building a quantum computer
\item Quantum sensing
\item Quantum information theory
\item Quantum machine learning
\item Several  machine learning/AI courses, at all levels
\end{enumerate}






\subsection{Strategy clarification}

\subsection{Resources, expertise, distribution of roles and cooperation}

\subsection{Project manager and project group}



\subsection{Budget}

{\em An accrual-based budget, cost plan and funding plan are to be entered
in the online grant application form. Fields are also provided there
for further specification and supplementary information.}

\subsection{Risk}

{\em Give an assessment of the risks related to the researcher school, both
in relation to the risk that the school cannot be implemented as
planned and the risk that the school does not achieve its objectives.}



\subsubsection*{Project organization and management}


\subsubsection*{BIBLIOGRAPHY}
\fontsize{10}{10}\selectfont{ \setlength{\bibsep}{1pt plus 0.2ex}
\bibliographystyle{apsrev}
\renewcommand{\bibsection}{\section*{}}
%\bibliography{qML20250301Zotero}
\end{document}
