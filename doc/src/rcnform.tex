\documentclass{scrreprt}
\usepackage[utf8]{inputenc}
\usepackage[T1]{fontenc}
\usepackage{graphicx}
\usepackage{longtable}
\usepackage{wrapfig}
\usepackage{rotating}
\usepackage[normalem]{ulem}
\usepackage{amsmath}
\usepackage{amssymb}
\usepackage{capt-of}
\usepackage{hyperref}
\usepackage[margin=2cm]{geometry}
\usepackage{setspace}
\usepackage[standardsections]{scrhack}
\usepackage{titlesec}
\usepackage{enumitem}
\usepackage{float}
\usepackage[export]{adjustbox}
\setlength{\parindent}{1em}
\setlength{\parskip}{.3em}
\usepackage[square,sort&compress,comma,numbers]{natbib}
\bibliographystyle{apsrev}
\usepackage[T1]{fontenc}
\usepackage{eurosym}
\usepackage{graphicx}
\graphicspath{{./figures/}}
\usepackage[labelformat=empty]{caption}
\renewcommand{\thechapter}{\arabic{chapter}}
\renewcommand{\thesection}{\arabic{section}}
\renewcommand{\thesubsection}{\thesection.\arabic{subsection}}
\renewcommand{\thesubsubsection}{\thesubsection.\arabic{subsubsection}}
\renewcommand{\chaptername}{TEST}
\titlespacing{\paragraph}{0pt}{4pt}{2pt}
\titlespacing{\subsection}{0pt}{6pt}{4pt}
\titlespacing{\section}{0pt}{4pt}{8pt}
\newcommand\fpm[1]{\textcolor{blue}{#1}}
\renewcommand{\bibsection}{\section*{}}

\author{Francesco Massel}
\date{\today}
\title{}
\hypersetup{
 pdfauthor={Francesco Massel},
 pdftitle={},
 pdfkeywords={},
 pdfsubject={},
 pdfcreator={Emacs 29.2 (Org mode 9.7.17)},
 pdflang={English}}
\begin{document}

\section*{\LARGE{Machine learning for quantum technology}}


\section{Excellence}
\label{sec:orgc59a8d0}
\textbf{Quantum technology (QT)} is an emerging field that takes advantage of fundamental quantum mechanical principles to develop advanced technological applications \cite{macfarlane_quantum_2003-1, acin_quantum_2018}. Its key domains are often categorized into four primary pillars: \textbf{quantum sensing, quantum communication, quantum computation, and quantum simulation}. Each of these pillars exploits the unique properties of quantum mechanics—such as superposition, entanglement, and coherence—to achieve tasks that are either impossible or highly inefficient for classical systems.
\subsection{State of the art, knowledge needs and project objectives}

\subsubsection*{State-of-the-art}

A fundamental challenge in quantum technologies is the detrimental
effect of environmental noise \cite{gardiner_quantum_2004}. Developing
error-resilient quantum technologies is therefore a critical milestone
in advancing quantum computing and precision metrology. Traditional
QEC protocols, such as surface codes \cite{fowler_surface_2012},
typically assume well-characterized noise models, limiting their
adaptability in practical scenarios. Similarly, quantum sensing
techniques often depend on predefined control strategies that do not
account for the complexity of real-world environments.


\subsubsection*{Project objectives \& knowledge needs.}


\subsection{Research questions and hypotheses, theoretical approach and methodology}


\subsection{Ethical issues}

Even if we fully recognize the importance of researchers engaging in
ethical considerations related to emerging and disruptive
technologies, such as QT and ML, we do not anticipate specific ethical
concerns arising from the particular research carried out in this
project. Although ethical reflections on the broader implications of
quantum technologies remain crucial, the scope of this project is
primarily technical and does not directly interface with contentious
ethical dilemmas such as privacy risks, security threats, or societal
disruptions.

However, given the increasing integration of ML techniques into QT, we
acknowledge that additional ethical concerns could emerge. These may
include issues related to interpretability. Furthermore, the
deployment of ML-based quantum sensing or error correction could raise
reliability and accountability questions, especially in high-stakes
applications.

\subsection{Novelty and ambition}

\subsubsection*{Novelty}

This project embodies a novel and ambitious approach by integrating RL
methodologies into quantum technologies, particularly in QEC and
QS. The ambition lies in co-designing strategies that not only
mitigate quantum noise but also leverage AI to dynamically adapt error
correction and sensing protocols.


\subsubsection*{Ambition}

  
\subsection{Measures for communication and exploitation}



\section{Implementation}
\label{sec:orgccfdc75}

\subsection{Project manager and project group}
\label{sec:orgc724ee7}

\subsubsection*{Project organization and management}
\label{sec:org4f1b7fc}

This project integrates expertise in QT and ML, structured into one administrative and four scientific work packages: \textbf{(WP0)} Project management, \textbf{(WP1)} Machine Learning for QEC, \textbf{(WP2)} Machine Learning for Quantum Sensing, \textbf{(WP3)} Algorithm Development and \textbf{(WP4)} Towards Experimental Validation.

\subsubsection*{BIBLIOGRAPHY}
\fontsize{10}{10}\selectfont{ \setlength{\bibsep}{1pt plus 0.2ex}
\bibliographystyle{apsrev}
\renewcommand{\bibsection}{\section*{}}
%\bibliography{qML20250301Zotero}
\end{document}
