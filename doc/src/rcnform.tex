\documentclass{scrreprt}
\usepackage[utf8]{inputenc}
\usepackage[T1]{fontenc}
\usepackage{graphicx}
\usepackage{longtable}
\usepackage{wrapfig}
\usepackage{rotating}
\usepackage[normalem]{ulem}
\usepackage{amsmath}
\usepackage{amssymb}
\usepackage{capt-of}
\usepackage{hyperref}
\usepackage[margin=2cm]{geometry}
\usepackage{setspace}
\usepackage[standardsections]{scrhack}
\usepackage{titlesec}
\usepackage{enumitem}
\usepackage{float}
\usepackage[export]{adjustbox}
\setlength{\parindent}{1em}
\setlength{\parskip}{.3em}
\usepackage[square,sort&compress,comma,numbers]{natbib}
\bibliographystyle{apsrev}
\usepackage[T1]{fontenc}
\usepackage{eurosym}
\usepackage{graphicx}
\graphicspath{{./figures/}}
\usepackage[labelformat=empty]{caption}
\renewcommand{\thechapter}{\arabic{chapter}}
\renewcommand{\thesection}{\arabic{section}}
\renewcommand{\thesubsection}{\thesection.\arabic{subsection}}
\renewcommand{\thesubsubsection}{\thesubsection.\arabic{subsubsection}}
\renewcommand{\chaptername}{TEST}
\titlespacing{\paragraph}{0pt}{4pt}{2pt}
\titlespacing{\subsection}{0pt}{6pt}{4pt}
\titlespacing{\section}{0pt}{4pt}{8pt}
\newcommand\fpm[1]{\textcolor{blue}{#1}}
\renewcommand{\bibsection}{\section*{}}

\author{Oslo Metropolitan University}
\date{\today}
\title{}
\hypersetup{
 pdfauthor={OsloMet},
 pdftitle={},
 pdfkeywords={},
 pdfsubject={},
 pdfcreator={Emacs 29.2 (Org mode 9.7.17)},
 pdflang={English}}
\begin{document}

\section*{\LARGE{Quantum TALENT}}





\section{Relevance to the call for the proposal}

The Quantum-TALENT (\underline{T}raining and \underline{A}dvanced
\underline{L}ectures in \underline{E}mergi\underline{N}g
\underline{T}echnologies) aims at developing a consistent national training and
educational program at a graduate  level in quantum technology, quantum
computing and quantum information theory and related topics within  artificial
intelligence/machine learning (AI/ML) methods for quantum science,
which will facilitate the development of a workforce with the
competences and knowledge to meet future technological challenges and
developments. This will  be done in close collaboration with all
partners and follows up with innovative educational initiatives  to the recent evaluation of Norwegian Physics.


\section{Excellence}

\subsection{Activities, organisation and cooperation}

\begin{enumerate}

\item Oslo Metropolitan University, Oslo, Norway

\item Department of Physics, University of Oslo

\item Kristiania University College, Oslo, Norway

\item University of South Eastern Norway, Kongsberg, Norway


\item Department of Physics, University of Bergen, Norway

\item Department of Physics, Norwegian University of Science and Technology, Trondheim, Norway

\item SIMULA research laboratory, Oslo, Norway

\end{enumerate}


Provide an expanded project description. Specify and describe the individual deliverables in the researcher school (activities, coordination activities, cooperation) in accordance with the requirements set out in the call. Please describe:
-	the expected scope, including the expected number of associated PhD candidates when the school is in full operation, as well as the assignment of supervisors and other scientific personnel;
-	the researcher school's scientific content in a broad sense, and objectives and delimitations, including the balance between measures to enhance scientific quality and relevance. The school’s activities should mainly be new initiatives or the continuation of existing initiatives. It must be clearly stated what is new;
-	national and (if relevant) international research groups and other partners that will form part of the network. Their contributions must be specified;
-	how the researcher school will quality assure activities and how the results will be documented.


Explain how the researcher school is linked to or will support the
doctoral degree programme at the collaborating institutions
scientifically, strategically and in terms of its benefit to
society. It is also important to clarify how the initiative is
innovative and will lead to added value beyond the activities that are
already in place or are currently being carried out.




\subsection{Activities, organization and cooperation}


\subsection{Background, context and needs}

{\em The primary and secondary objectives of the researcher school must be
specified in the online grant application form.  Describe the
background for the application and explain why the researcher school
and its activities are needed. Make it clear how the project is part
of larger systems and will build upon and reinforce these
(context). It is important to document good knowledge of relevant
research areas, existing structures and activities, and the central
challenges in the field, both nationally and internationally.}





A fundamental challenge in quantum technologies is the detrimental
effect of environmental noise \cite{gardiner_quantum_2004}. Developing
error-resilient quantum technologies is therefore a critical milestone
in advancing quantum computing and precision metrology. Traditional
QEC protocols, such as surface codes \cite{fowler_surface_2012},
typically assume well-characterized noise models, limiting their
adaptability in practical scenarios. Similarly, quantum sensing
techniques often depend on predefined control strategies that do not
account for the complexity of real-world environments.

\subsection{Added value}

{\em Explain how the researcher school is linked to or will support the
doctoral degree programme at the collaborating institutions
scientifically, strategically and in terms of its benefit to
society. It is also important to clarify how the initiative is
innovative and will lead to added value beyond the activities that are
already in place or are currently being carried out.}

\section{Impact}


{\em Describe the impacts and outcomes expected from the researcher school
in both the short-term and the medium-term, and the
societal/system-related impacts the school will help to generate in a
longer perspective.}

\subsection{Communication, dissemination and target groups}

{\em Specific plans for communication, including advertisement/marketing,
target groups, relevant user groups and how these will be involved in
the researcher school, which channels are to be used etc. is to be
provided in the online application form.}

\subsection{Relevance to society}


\subsection{Environmental impact, ethical perspectives, recruitment of women/gender balance }

{\em Briefly describe whether implementing the researcher school and/or applying the results will have a significant positive or negative environmental impact. Clarify whether there are any ethical issues relating to implementation of the researcher school. If there are, briefly describe how these will be dealt with. Provide a brief explanation of how the researcher school will promote the Research Council’s general objectives to increase the recruitment of women in research and improve the gender balance.}



Even if we fully recognize the importance of researchers engaging in
ethical considerations related to emerging and disruptive
technologies, such as QT and ML, we do not anticipate specific ethical
concerns arising from the particular research carried out in this
project. Although ethical reflections on the broader implications of
quantum technologies remain crucial, the scope of this project is
primarily technical and does not directly interface with contentious
ethical dilemmas such as privacy risks, security threats, or societal
disruptions.



\section{Implementation}

These courses address many central concepts, such as quantum
mechanical superposition, entanglement, QT applications, and many
different methods and algorithms for AI and machine learning, covering
both supervised and unsupervised learning as well as central
discriminative and generative deep learning methods. Programming is
indispensable in all courses and course participants learn to
study complicated problems which require knowledge and skills
necessary for educating a modern workforce. Many of these courses
focus also on teamwork, project management, and communication.



\begin{enumerate}
\item Review of Quantum physics, with elements from linear algebra and statistics
\item Basic introduction to QT, QIT and QC. All involved universities offer similar courses
\item Quantum mechanics (only at UiO as of now)
\item Quantum Computing and central algorithms, gates, circuits and more
\item Building a quantum computer
\item Quantum sensing
\item Quantum information theory
\item Quantum machine learning
\item Several  machine learning/AI courses, at all levels
\end{enumerate}






\subsection{Strategy clarification}

\subsection{Resources, expertise, distribution of roles and cooperation}

\subsection{Project manager and project group}



\subsection{Budget}

{\em An accrual-based budget, cost plan and funding plan are to be entered
in the online grant application form. Fields are also provided there
for further specification and supplementary information.}

\subsection{Risk}

{\em Give an assessment of the risks related to the researcher school, both
in relation to the risk that the school cannot be implemented as
planned and the risk that the school does not achieve its objectives.}



\subsubsection*{Project organization and management}


\subsubsection*{BIBLIOGRAPHY}
\fontsize{10}{10}\selectfont{ \setlength{\bibsep}{1pt plus 0.2ex}
\bibliographystyle{apsrev}
\renewcommand{\bibsection}{\section*{}}
%\bibliography{qML20250301Zotero}
\end{document}
