%\documentclass[compress]{beamer}
\documentclass{beamer}
\colorlet{structure}{green!50!black}
\mode<article> % only for the article version
{
  \usepackage{beamerbasearticle}
  \usepackage{fullpage}
  \usepackage{hyperref}
}
\beamertemplateshadingbackground{red!10}{blue!10}
\setbeamertemplate{footline}[page number]
\usepackage{beamerthemeshadow}
\usepackage{pgf,pgfarrows,pgfnodes,pgfautomata,pgfheaps,pgfshade}
\usepackage{graphicx}
\usepackage{amsmath,amssymb}
\usepackage[latin1]{inputenc}
\usepackage{colortbl}
\usepackage[english]{babel}
% Use some nice templates
\beamertemplatetransparentcovereddynamic
\newcommand{\be}{\begin{equation}}
\newcommand{\ee}{\end{equation}}
\newcommand{\bra}[1]{\left\langle #1 \right|}
\newcommand{\ket}[1]{\left| #1 \right\rangle}
\newcommand{\braket}[2]{\left\langle #1 \right| #2 \right\rangle}
\newcommand{\OP}[1]{{\bf\widehat{#1}}}
\newcommand{\matr}[1]{{\bf \cal{#1}}}
\def\d{\text{d}}
\def\sech{\text{sech}}
\def\Cdot{\!\cdot\!} 
\def\e{\text{e}}
\def\L_2{\L_{2}}
\def\fm{\:\:\mathrm{fm}^{-1}}

%\newcommand{\braket}[1]{\langle#1\rangle}
%\newcommand{\Span}{\operatorname{sp}}
%\newcommand{\tr}{\operatorname{trace}}
%\newcommand{\diag}{\operatorname{diag}}

\newcommand{\pdiff}[2]{\frac{\partial#1}{\partial#2}}
\newcommand{\wedgeprod}[2]{\overset{#2}{\underset{#1}{\wedge}}}
\newcommand{\directsum}[2]{\overset{#2}{\underset{#1}{\bigoplus}}}
\newcommand{\Span}{\operatorname{span}}
\newcommand{\Hilb}{{\mathcal{H}}}
\newcommand{\Proj}{{\mathcal{P}}}
\newcommand{\Model}{{\mathcal{M}}}
\newcommand{\CalS}{{\mathcal{S}}}
\newcommand{\PhiFD}{\Phi^\mathrm{FD}}
\newcommand{\Ind}{{\mathcal{I}}}
\newcommand{\rmd}{\mathrm{d}}
\newcommand{\sgn}{\operatorname{sgn}}
\newcommand{\eofproof}{\hspace{1em}$\diamondsuit$}

\usetheme{Copenhagen}

\title{Potential national courses in quantum technology, quantum computing and quantum information theory}

\author{Gabriel Balaban, Marianne Bathen, Morten Hjorth-Jensen, Lasse Vines, Francesco P. Massel, S\o lve Selst\o, Susanne Viefers}
\institute{Kristiania University College, OsloMet University, University of Oslo, University of South-Eastern Norway}

\date{February 2025}

\begin{document}

\frame{\titlepage}


\part<presentation>{Main Talk}


\begin{frame}[plain,fragile]
\frametitle{Education and knowledge in quantum science through the development of an advanced national educational program}

The Quantum-TALENT (\underline{T}raining and \underline{A}dvanced
\underline{L}ectures in \underline{E}mergi\underline{N}g
\underline{T}echnologies) aims at developing a consistent training and
educational program at all levels in quantum technology, quantum
computing and quantum information theory and related topics within  artificial
intelligence/machine learning (AI/ML) methods for quantum science,
which will facilitate the development of a workforce with the
competences and knowledge to meet future technological challenges and
developments. This could  be done in close collaboration with all
partners.
\end{frame}

\begin{frame}[plain,fragile]
\frametitle{Practical implementations}


\begin{block}{}
\begin{enumerate}
\item The courses could be listed (in order to enhance visibility) at every institution with own course codes.
\item Students enlist at their respective institutions for credit accreditation, needs national coordination
\item Courses can be modularized and taught intensively
\item Material shared by all institutions
\item Taught by teachers (experts) at the different institutions
\item Courses can be taught fully at one institution as regular one-semester courses or taught as intensive courses
\item Online and in-person attendance
\item Courses can be taught as regular courses at each institution
\end{enumerate}
\end{block}
\end{frame}

\begin{frame}[plain,fragile]
\frametitle{Acronyms}


\begin{block}{}
\begin{enumerate}
\item QIT= Quantum information theory
\item QC= Quantum computing
\item QT=Quantum technology
\item ML = Machine learning
\item AI= Artificial intelligence
\end{enumerate}
\end{block}
\end{frame}



\begin{frame}[plain,fragile]
\frametitle{Quantum-TALENT: Education for a broader audience}

We have yearslong experience (with research based evidence on what works or not) in developing intensive training courses on ML/AI and QT/QIT/QC.

\begin{block}{}
\begin{enumerate}
\item Intensive short courses on selected topics (which can lead to credits and certificates)

\item Certificates of expertise with modules that can add up to one year of credits or more.

\item One-year study (\aa rstudium) in QIT/QC/QT and AI/ML


\item Possibilities of adding up to a master specialization in quantum science/technologies with the addition of AI/ML methods

\item Common educational projects and supervision of students
\end{enumerate}

\noindent
\end{block}
\end{frame}




\begin{frame}[plain,fragile]
\frametitle{Universities and academic partners}

\begin{block}{}
\begin{enumerate}
\item Department of Physics, University of Oslo

\item Department of Chemistry, University of Oslo

\item Center for Computing in Science Education, University of Oslo

\item Center for Materials Science and Nanotechnology, University of Oslo

\item Hylleraas Center for Quantum Molecular Sciences, University of Oslo

\item Kristiania University College, Oslo, Norway

\item University of South Eastern Norway, Kongsberg, Norway

\item Oslo Metropolitan University 

\item SIMULA research laboratory, planned but not confirmed
\end{enumerate}

\noindent
\end{block}
\end{frame}

\begin{frame}[plain,fragile]
\frametitle{Possible courses that can be offered, courses can be modularized}

\begin{block}{}
\begin{enumerate}
\item Review of Quantum physics, with elements from linear algebra and statistics
\item Basic introduction to QT, QIT and QC. All involved universities offer similar courses
\item Quantum mechanics (only at UiO as of now)
\item Quantum Computing and central algorithms, gates, circuits and more
\item Building a quantum computer
\item Quantum sensing
\item Quantum information theory
\item Quantum machine learning
\item Several  machine learning/AI courses, at all levels
\end{enumerate}

\end{block}
\end{frame}

\begin{frame}[plain,fragile]
\frametitle{What we have and could be offered as pilots}

\begin{block}{}
\begin{enumerate}
\item Introduction to Quantum Technology. These courses have been developed and are taught at  all institutions as of now)
\item Quantum computing offered at UiO (as three courses, see list below) and  OsloMet. The UiO courses have been taught already as intensive courses at different schools. 
\item Quantum physics and quantum mechanics, only at UiO
\item Introduction to machine  learning and advanced machine learning, courses at UiO. These have already been taught as intensive courses at several schools.
\end{enumerate}

\end{block}
\end{frame}


\begin{frame}[plain,fragile]
\frametitle{Structure of courses}

These courses address many central concepts, such as quantum
mechanical superposition, entanglement, QT applications, and many
different methods and algorithms for AI and machine learning, covering
both supervised and unsupervised learning as well as central
discriminative and generative deep learning methods. Programming is
indispensable in all courses and course participants learn to
study complicated problems which require knowledge and skills
necessary for educating a modern workforce. Many of these courses
focus also on teamwork, project management, and communication.
\end{frame}



\begin{frame}[plain,fragile]
\frametitle{Existing programs/courses; Kristiania University College}

\end{frame}

\begin{frame}[plain,fragile]
\frametitle{Existing programs/courses; OsloMET university}

\end{frame}


\begin{frame}[plain,fragile]
\frametitle{Existing programs/courses; University of South-Eastern Norway}

\end{frame}


\begin{frame}[plain,fragile]
\frametitle{Existing programs, UiO}

At the university of Oslo we have now established several educational
programs in AI and QTs and quantum science. These programs span the
whole spectrum from beginners courses to advanced training and
education tailored to the specific needs of the participants.
\end{frame}

\begin{frame}[plain,fragile]
\frametitle{Study program addressing Quantum Technologies and AI}

\begin{block}{}
\begin{enumerate}
\item The Physics and Astronomy bachelor program \href{{https://www.uio.no/studier/program/fysikk-astronomi}}{\nolinkurl{https://www.uio.no/studier/program/fysikk-astronomi}} at the University of Oslo offers from fall 2024 a study direction in \textbf{Quantum Technologies}

\item The Master of Science program in \href{{https://www.uio.no/english/studies/programmes/computational-science-master/}}{\nolinkurl{https://www.uio.no/english/studies/programmes/computational-science-master/}} Computational Science offers a study direction in Quantum Science and Technology in the Computational Science program, start fall 2025

\item The Master of Science program in Physics offers a study direction in Quantum Technologies and Quantum Science, start fall 2025

\item PhD research on AI/machine learning, quantum technologies and quantum science

\item In addition there are many introductory and advanced courses in machine learning and AI.
\end{enumerate}

\noindent
\end{block}
\end{frame}


\begin{frame}[plain,fragile]
\frametitle{Courses at UiO}

\begin{block}{}
\begin{enumerate}
\item FYS1400--Introduction to Quantum Technology
\item FYS3405--Kondenserte fasers fysikk og kvantematerialer
\item FYS3415--Kvantedatamaskiner og kvanteinformasjon
\item FYS5419--Quantum computing and quantum machine learning
\item MAT3420--Quantum Computing
\item MAT4430--Quantum information theory
\item FYS-STK4155--Applied Data Analysis and Machine Learning
\item FYS5429--Advanced machine learning and data analysis for the physical sciences
\end{enumerate}

\end{block}
\end{frame}






\end{document}


